\section{Glossary}
\writer{Georgios}

\subsection{Technical Glossary}
\begin{description}
  \item[Animation] The dynamic behavior of objects on 3D Canvas that are associated with Places of the Petri Net model.
  \item[Appearance] The look and feel of the objects visualized by the 3DEngine .
  \item[Appearance Editor] A graphical editor provided EMF for configuring appearance.
  \item[Control GUI] A simple Graphical User Interface with appropriate actions (play/pause/stop buttons) so as the end user to be able to interact with the 3DEngine and consequently with underlying Petri Net execution .
  \item[Geometry] the track (3D line/curve) on which the 3D-visualization of the Petri Net simulation takes place. The geometry consists of predefined objects such as, semicircles, points and lines.
  \item[Geometry Editor] A graphical editor provided by ePNK and GMF for designing the area on which the simulation will be reflected.
  \item[Identity] The attribute of Arcs in the Petri Net model that defines the trajectory of the Tokens.
  \item[IgnoreAnimation] The attribute of Arcs in the Petri Net model that allows the target transition to fire without the associated Place Animation having finished.
  \item[Input place] a place in which tokens can be insterted externally, i.e. by clicking an interactive control point.
  \item[Interactive control point] A point that allows the user to interact with the system (e.g. by a click)
  \item[Petri Net Editor] A graphical editor provided by ePNK and GMF for designing the Petri Net of the system a user wants to visualize.
  \item[Physical object] A graphical object that is used by the 3DEngine for visualizing its behaviour during the Petri Net simulation.
  \item[Simulator] The software component that will provide the underlying Petri Net execution information to the 3DEngine so as to be able to visualize the Petri Net execution with 3D objects.
  \item[Shape] The visual traits of a physical object being part of the simulated system visualization.
  \item[Token] The Petri Net element that moves along Petri Net places through transitions.
\end{description}

\subsection{Technology Terms}
\begin{description}
\item[Ecore Tool editor] - The EMF editor that provides all the necessary elements for realizing or drawing the model in question.
\item[EMF] - The Eclipse Modeling Framework. It auto-generates code that represents the corresponding model (usually described in UML).
\item[EMF Validation Framework OCL Integration] -  Helps for expressing constraints on ambiguous graphical models such as a class diagram. Tightly used with UML users can define constraints on their models.
\item[ePNK] - The eclipse Petri Net  Kernel, a modern equivalent of PNVis missing the visualization part though. Built following the model-based software engineering paradigm. ePNK Petri Net Types to be extended to provide new functionality.
\item[GMF] - The Eclipse Graphical Modeling Framework provides the appropriate infrastructure for developing graphical editors based on EMF.
\item[Xtext] - Xtext is a framework for development of programming languages and domain specific languages.
\end{description}
