\section{Introduction}
\writer{Marius}

This report documents the requirements of \epns, a software system that is being developed in 02162 Software Engineering 2 course.

\epns aims to create an easy way to visualize and interact with a simulation for a continuous dynamic system
(for example, a railway system).
The dynamic system will be modeled by a Petri net. To allow the modeling of continuous systems,
the classical concepts of Petri nets are extended so that the new version can model more than discrete systems.


\subsection{Conventions}

The requirements specified in this document will be organized, according to their priority,
by the following three keywords: shall, should and would be nice.
\begin{description}
  \item[shall] - describes the basic requirements, which allows the system to
  work properly as desired
  \item[should]  - describes the requirements which add functionalities that are
  not mandatory for the system to run, but would be important
  \item[would be nice] - describes the requirements that are good ideas, but
  have the lowest priority; they will be implemented only if the resources allow it or in further versions of the software
\end{description}

\subsection{Audience}
 
The audience of this document is represented by people who are interested in Petri nets, persons affiliated with companies that model systems using Petri nets and users of \epns.

